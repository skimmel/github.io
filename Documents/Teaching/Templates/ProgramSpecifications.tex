\documentclass[11pt,landscape]{article}

\usepackage[margin=1in]{geometry}
\usepackage{enumitem}
\usepackage[colorlinks]{hyperref}
\usepackage{url}\urlstyle{same}
\setlist[itemize]{noitemsep, topsep=0pt, leftmargin=*,partopsep=0pt, parsep=0pt,after=\strut}
\usepackage{amsfonts}
\usepackage{multicol}
\usepackage{array}
\pagestyle{empty}
\setlength{\parindent}{0pt}
\setlength{\parskip}{3pt}

\begin{document}

\hfill Shelby Kimmel

\begin{center}
{\huge Grading Rubric for Programming Problems}
\end{center}
\bigskip

In order to get credit for a programming assignments, it must meet the following five criteria:

\begin{itemize}
\item \textbf{Program Specifications} You program must work correctly on all inputs. Also, if there any specifications about how the program should be written, or how the output should appear, those specifications should be followed.

\item \textbf{Readability} Variables and functions should have meaningful names. Code should be organized into functions/methods where appropriate. There should be an appropriate amount of white space so that the code is readable, and indentation should be consistent.

\item \textbf{Documentation} Your code and functions/methods should be appropriately commented. However, not every line should be commented because that makes your code overly busy. Think carefully about where comments are needed. 

\item \textbf{Code Efficiency} You program should not contain needlessly hard-coded or repeated code. (It does not have to have optimal asymptotic (big-O) efficiency.

\item \textbf{Assignment Specifications} The assignment will likely ask you to include certain information as comments, or save your program with a certain file name, or other such specifications. You must follow these guidelines.
\end{itemize}







\end{document}