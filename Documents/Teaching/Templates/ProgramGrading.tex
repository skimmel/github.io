\documentclass[11pt,landscape]{article}

\usepackage[margin=1in]{geometry}
\usepackage{enumitem}
\usepackage[colorlinks]{hyperref}
\usepackage{url}\urlstyle{same}
\setlist[itemize]{noitemsep, topsep=0pt, leftmargin=*,partopsep=0pt, parsep=0pt,after=\strut}
\usepackage{amsfonts}
\usepackage{multicol}
\usepackage{array}
\pagestyle{empty}
\setlength{\parindent}{0pt}
\setlength{\parskip}{3pt}

\begin{document}

\hfill Shelby Kimmel

\begin{center}
{\huge Grading Rubric for Programming Problems}
\end{center}
\bigskip

Programming assignments will be graded based on a 32-point rubric (see next page). On this page, we give a little more information about the categories in the rubric:

\begin{itemize}
\item \textbf{Rough Draft:} You should submit code that runs and that implements or sets up a small part of the larger assignment.

\item \textbf{Program Correctness:} You program should work correctly on all inputs. Also, if there any specifications about how the program should be written, or how the output should appear, those specifications should be followed.

\item \textbf{Readability:} Variables and functions should have meaningful names. Code should be organized into functions/methods where appropriate. There should be an appropriate amount of white space so that the code is readable, and indentation should be consistent.

\item \textbf{Documentation} Your code and functions/methods should be appropriately commented. However, not every line should be commented because that makes your code overly busy. Think carefully about where comments are needed. 

\item \textbf{Code Elegance} There are many ways to write the same functionality into your code, and some of them are needlessly slow or complicated. For example, if you are repeating the same code, it should be inside  creating a new method/function or for loop.

\item \textbf{Assignment Specifications} The assignment will likely ask you to include certain information as comments, or save your program with a certain file name, or other such specifications. These tasks fall under ``assignment specifications.''
\end{itemize}


{\renewcommand{\arraystretch}{2}
\begin{tabular}{|p{3.3cm}|p{4.2cm}|p{4.2cm}|p{4.2cm}|p{4.2cm}|}
\hline
{\bf Rough Draft \newline Correctness } & & {\bf 2 points} & {\bf 1 points} & {\bf 0 point} \\
\hline
& 
& 
Rough draft runs and implements some part of the assignment.& 
Rough draft submitted, but does not run.& 
No rough draft submitted.\\
\hline
{\bf Program \newline Correctness } & {\bf 15 points} & {\bf 10 points} & {\bf 5 points} & {\bf 0 point} \\
\hline
& 
 Program always works correctly and meets the specifications& 
 Minor details of the program specification are violated, program functions incorrectly on some inputs.& 
 Significant details of specification are violated, or the program often exhibits incorrect behavior.& 
 Program does not compile, or errors occur on input similar to sample.\\
\hline
 \end{tabular}
 }\\
 {\renewcommand{\arraystretch}{2}
\begin{tabular}{|p{3.3cm}|p{4.2cm}|p{4.2cm}|p{4.2cm}|p{4.2cm}|}
\hline
{\bf Readability } & {\bf 6 points} & {\bf 4 points} & {\bf 2 points} & {\bf 0 point} \\
\hline

& 
Code is clean, understandable, well-organized &
Minor issues such as inconsistent indentation, variable naming, general organization & 
At least one major issue that makes it difficult to read & 
Several major issues that make it difficult to read.\\
\hline
{\bf Documentation} & {\bf 3 points}& {\bf 2 points} & {\bf 1 points} & {\bf 0 point} \\
\hline
 &
Code is well commented. 
&
One or two places could benefit from comments, or the code is overly commented
&
Major lack of comments make it difficult to understand code.
&
No comments.\\
\hline
{\bf Code Elegance } & & {\bf 4 points} & {\bf 2 points} & {\bf 0 point} \\
\hline
& &
Code appropriately uses for loops and methods for repeated code, and there is minimal hard-coding. &
Code uses a poorly chosen approach in at least one place, for example, hard coding something that could be implemented through a \texttt{for} loop & 
Many instances where code could have used easier/faster/better approach.\\
\hline
{\bf Assignment specifications} & & {\bf 2 points} & {\bf 1 points} & {\bf 0 point} \\
\hline
 & &
Assignment meets specifications
&
Minor specifications are violated
&
Significant specifications ignored or violated
\\
\hline
 \end{tabular}
}






\end{document}