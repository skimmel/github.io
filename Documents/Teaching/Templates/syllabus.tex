\documentclass[11pt]{article}

\usepackage[margin=1in]{geometry}
\usepackage{enumitem}
\usepackage[colorlinks]{hyperref}
\usepackage{url}\urlstyle{same}

\pagestyle{empty}
\setlength{\parindent}{0pt}
\setlength{\parskip}{3pt}

\begin{document}

% ~\vspace{-4em}
% \hrule\hrule\smallskip\centerline{\Huge{\bf DRAFT}}\smallskip\hrule\hrule
% \bigskip

{\Large {Course Number} \hfill Semester} \\[3pt]
{\Large {\bf Math Course Title}}

\bigskip

{\bf Course website} \url{Here}

\medskip

{\bf Learning Goals}

After completing this course, students should be able to
\begin{enumerate}
\item Use mathematical notation to describe and reason about basic data types, such as sets, numbers, and graphs.
\item Write formal proofs using techniques such as induction, contradiction, counterexample, exhaustive
\item Analyze basic computational protocols using these mathematical techniques.
\end{enumerate}

{\bf Overview}

A quantum mechanical representation of information allows one to efficiently perform certain tasks that are intractable within a classical framework.  This course aims to give a basic foundation in the field of quantum information processing.  Students will be prepared to pursue further study in quantum computing, quantum information theory, and related areas.  No previous background in quantum mechanics is required.

\medskip

{\bf Course topics}

Algorithmic Paradigms:
\begin{itemize}
\item Divide and Conquer
\item Greedy Algorithms
\item Dynamic Programming
\end{itemize}

Guiding Principles:
\begin{itemize}
\item Worst case
\item Big-O
\item Asymptotic!!
\end{itemize}


Problem: merge-sort plus insertion sort at small problem sizes


Basic model of quantum computation (reversible computing, qubits, unitary transformations, measurements, quantum protocols, quantum circuits); quantum algorithms (simple query algorithms, the quantum Fourier transform, Shor's factoring algorithm, Grover's search algorithm and its optimality); quantum complexity theory; mixed quantum states and quantum operations; quantum information theory (entropy, compression, entanglement transformations, quantum channel capacities); quantum error correction and fault tolerance; quantum nonlocality; quantum cryptography (key distribution and bit commitment); selected additional topics as time permits.

For a detailed lecture schedule with recommended readings, see the \href{http://terp.ps/introqip}{course website}.

\medskip

{\bf Prerequisites}

Familiarity with complex numbers and basic concepts in linear algebra (e.g., eigenvalues, eigenvectors, Hermitian and unitary matrices) is required.  Students are not expected to have taken previous courses in quantum mechanics or the theory of computation.

\medskip

{\bf Coordinates}

Tuesday/Thursday, 12:30 am--1:45 pm, CSI 3120

\medskip

{\bf Instructors}

Andrew Childs (\href{mailto:amchilds@umd.edu}{amchilds@umd.edu}) \\
Office hours: Starting Sept 21, Wed 3:30-4:30 pm (CSS 3100F)

\smallskip

Shelby Kimmel (\href{mailto:shelbyk@umd.edu}{shelbyk@umd.edu}) \\
Office hours: Tues 4:00-5:00 pm (CSS 3100E)

\smallskip

Brad Lackey (\href{mailto:bclackey@umd.edu}{bclackey@umd.edu}) \\
Office hours: Mon 2:00-3:00 pm (CSS 3100G) 

%\smallskip
%AVW Office Hours (manned by that week's lecturer)\\
%Office hours: Tuesday 2:00-3:00 pm AVW 3225

\medskip

{\bf Teaching assistant}

Yuan Su (\href{mailto:yuansu@umd.edu}{yuansu@umd.edu}) \\
Office hours: Tuesday 3:30-4:30 pm (AVW 3225)

\medskip 

{\bf Additional Office Hours}

AVW 3225 on Tuesdays from 2-3 pm, held by whoever is lecturing that day

\newpage

{\bf Texts}

Primary: Paul Kaye, Raymond Laflamme, and Michele Mosca, \emph{An Introduction to Quantum Computing}, Oxford University Press (2007).

Supplemental: Michael A.\ Nielsen and Isaac L.\ Chuang, \emph{Quantum Computation and Quantum Information}, Cambridge University Press (2000).

Copies of both texts will be available on reserve in the Engineering and Physical Sciences Library (Math building, room 1403).

\medskip

% \clearpage
{\bf Evaluation}

Your final grade will be determined as follows:

\begin{tabular}{@{}lr@{}l}
Assignments &  8\% & ~each (40\% total) \\
Project     & 30\% \\
Final exam  & 30\%

\end{tabular}

\medskip

{\bf Assignments}

There will be 5 homework assignments during the course.  Assignments will be made available on the \href{http://ter.ps/introqip}{course website} and will be due at the start of class.  Solutions will be posted on the course website soon after the due date, so extensions will not be granted.  Graded assignments will be returned in class.

You are encouraged to discuss homework problems with your peers, with the TA, and with the course instructor.  However, your solutions should be based on your own understanding and should be written independently.  For each assignment, you must either include a list of students in the class with whom you discussed the problems, or else state that you did not discuss the assignment with your classmates.

\medskip

{\bf Project}

Students will be expected to write an expository paper on a topic of their choice from the quantum information literature. Further details about the scope of the paper, submission guidelines, and a list of possible project topics will be posted on the \href{http://ter.ps/introqip}{course website}.

As part of this project, you will work with a partner to review drafts of your individual papers.  The timeline and grading rubric for the project are as follows:
% [Note about deadline extensions?]
\begin{itemize}[topsep=0pt,itemsep=-3pt]
\item \emph{October 13}: Proposal and outline of paper submitted to instructors. \newline
[5\%: based on timely completion and reasonable effort]
\item \emph{November 10}: Rough draft submitted to instructors and peer reviewer. \newline
[5\%: based on timely completion and reasonable effort]
\item \emph{November 14}: Critique of partner's draft submitted to instructors and peer reviewer. \newline
[5\%: based on timely completion and reasonable effort]
\item \emph{November 15--22}: Meet with partner and an instructors to discuss both partners' rough drafts. \newline
[5\%: based on timely completion and reasonable effort]
\item \emph{December 8}: Final draft submitted to instructors. \newline
[80\%: 40\% for scientific content, 40\% for clarity of exposition]
\end{itemize}
Please respect these deadlines to facilitate the peer review process and to receive full credit.

\medskip

{\bf Final exam}

The course will include a take-home final exam.  The exam will be made available on the morning of Wednesday, December 14, and will be due by 4 pm on Friday, December 16.  Students may choose to take the exam during any three-hour period during that time.

\medskip

{\bf Academic accommodations}

Any student eligible for and requesting reasonable academic accommodations due to a disability is asked to provide, to the instructor during office hours, a letter of accommodation from the Office of Disability Support Services (DSS) within the first two weeks of the semester.

If you plan to observe any holidays during the semester that are not listed on the university calendar, please provide a list of these dates by the end of the first week of the semester.

As mentioned above, extensions to assignment due dates will not be granted for any reason, so that all students can have timely access to solutions.  In the event of a medical emergency that affects your ability to complete coursework, appropriate accommodations will be made.  However, you must make a reasonable attempt to notify the instructor prior to the due date, and you must provide written documentation from the Health Center or an outside health care provider.  This documentation must verify dates of treatment and indicate the timeframe that you were unable to meet academic responsibilities. It must also contain the name and phone number of the medical service provider in case verification is needed. No diagnostic information will ever be requested.

\medskip

{\bf Course evaluations}

Course evaluations are an important part of evaluating instruction.  The Department of Computer Science and its faculty take student feedback seriously.  Students can go to \url{http://www.courseevalum.umd.edu} to complete their evaluations.

\end{document}