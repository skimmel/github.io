\documentclass[12pt]{article}
%\usepackage{geometry} % to change the page dimensions
%\geometry{margin=1in} % for example, change the margins to 2 inches all round

%%%%%%%%%%%%%%%%%%%%%%%%%
%  N E W T H E O R E M  %
%%%%%%%%%%%%%%%%%%%%%%%%%

\usepackage{amsmath}
\usepackage{fancyhdr}
\usepackage{amssymb}
\usepackage{amsthm}
\usepackage{graphicx}
\usepackage{varioref}
\usepackage{verbatim} 
\usepackage{multicol}
\usepackage{enumerate}
\usepackage[normalem]{ulem}
%\usepackage[margin=1in]{geometry}
\usepackage{caption}
\usepackage{subcaption}
\usepackage[T1]{fontenc}
\usepackage[margin=1in]{geometry}


\usepackage{mathrsfs}

\usepackage{url}\urlstyle{same}
\usepackage{xspace}
\usepackage{thm-restate}

% Nicer font
\usepackage{mathpazo}


% Microtype
\usepackage{microtype}

% TikZ
\usepackage{tikz}
\usetikzlibrary{calc}

% Support eps figures
\usepackage{epstopdf}

% Hypertext package
\usepackage[colorlinks = true]{hyperref}
% Title and authors
%\hypersetup{
%  pdftitle = {},
%  pdfauthor = {}
%}
% Color definitions
\usepackage{xcolor}
\definecolor{darkred}  {rgb}{0.5,0,0}
\definecolor{darkblue} {rgb}{0,0,0.5}
\definecolor{darkgreen}{rgb}{0,0.5,0}
% Color links
\hypersetup{
  urlcolor   = blue,         % color of external links
  linkcolor  = darkblue,     % color of internal links
  citecolor  = darkgreen,    % color of links to bibliography
  filecolor  = darkred       % color of file links
}

% Clever references
\usepackage{cleveref}%[nameinlink]
\crefname{lemma}{Lemma}{Lemmas}
\crefname{proposition}{Proposition}{Propositions}
\crefname{definition}{Definition}{Definitions}
\crefname{theorem}{Theorem}{Theorems}
\crefname{conjecture}{Conjecture}{Conjectures}
\crefname{corollary}{Corollary}{Corollaries}
\crefname{section}{Section}{Sections}
\crefname{appendix}{Appendix}{Appendices}
\crefname{figure}{Fig.}{Figs.}
\crefname{equation}{Eq.}{Eqs.}
\crefname{table}{Table}{Tables}
\crefname{claim}{Claim}{Claims}

%%%%%%%%%%%%%%%%%%%%%%%%%
%  N E W T H E O R E M  %
%%%%%%%%%%%%%%%%%%%%%%%%%

\newtheorem{theorem}{Theorem}
\newtheorem{lemma}[theorem]{Lemma}
\newtheorem{proposition}[theorem]{Proposition}
\newtheorem{definition}[theorem]{Definition}
\newtheorem{corollary}[theorem]{Corollary}
\newtheorem{conjecture}[theorem]{Conjecture}
\newtheorem*{conjecture*}{Conjecture}
\newtheorem*{problem}{Problem}
\newtheorem{claim}[theorem]{Claim}
\theoremstyle{definition}
\newtheorem*{remark}{Remark}
\newtheorem*{example}{Example}


%%%%%%%%%%%%%%%%%%%
%  Comments  %
%%%%%%%%%%%%%%%%%%%
\newcommand{\todo}[1]{{\color{red}{[{\bf TODO:} #1]}}}
\newcommand{\skimmel}[1]{{\color{violet}{[{\bf skimmel:} #1]}}}
\newcommand{\tocite}[1]{{\color{blue}{[{\bf CITE:}#1]}}}
\newcommand{\tocheck}[1]{{\color{red}{[{\bf TO CHECK:}#1]}}}


%%%%%%%%%%%%%%%%%%%
%  Author Affiliations  %
%%%%%%%%%%%%%%%%%%%
\usepackage{authblk}
\renewcommand\Authfont{\scshape}
\renewcommand\Affilfont{\itshape\small}



%%%%%%%%%%%%%%%%%%%
%  New Commands  %
%%%%%%%%%%%%%%%%%%%
\newcommand\QMA{{\sf{QMA}}}
\newcommand\QCMA{{\sf{QCMA}}}
\newcommand\NP{{\sf{NP}}}

\newcommand{\ket}[1]{|#1\rangle}
\newcommand{\bra}[1]{\langle#1|}
\newcommand{\proj}[1]{|#1\rangle\!\langle#1|}

\renewcommand{\th}[1]{${#1}^{\textrm{th}}$}
\newcommand{\tth}[0]{\textsuperscript{th}}
\newcommand{\st}[0]{\textsuperscript{st}}
\newcommand{\nd}[0]{\textsuperscript{nd}}
\newcommand{\rd}[0]{\textsuperscript{rd}}

\DeclareMathAlphabet{\matheu}{U}{eus}{m}{n}

\DeclareMathOperator{\tr}{tr}
\newcommand{\Paulis}{{\matheu P}}
\newcommand{\Clifs}{{\matheu C}}
\newcommand{\Hilb}{{\matheu H}}
\newcommand{\sop}[1]{{\mathcal #1}}
\newcommand{\PL}[1]{{#1}^{P\!L}}
\newcommand{\I}{{\mathbb I}}
\newcommand{\CNOT}{{\mathrm{CNOT}}}
\newcommand{\plr}[1]{\hat{#1}} %pauli-liouville-represenation


\newcommand{\braket}[2]{\langle{#1}|{#2}\rangle}
\newcommand{\ketbra}[2]{|{#1}\rangle\!\langle{#2}|}
\newcommand{\kket}[1]{|{#1}\rangle\!\rangle}
\newcommand{\bbra}[1]{\langle\!\langle{#1}|}
\newcommand{\bbrakket}[2]{\langle\!\langle{#1}|{#2}\rangle\!\rangle}


\newcommand{\ceil}[1]{\left\lceil{#1}\right\rceil}

%-----------------------------------------------------------------------------%




\begin{document}
\hfill Shelby Kimmel

\begin{center}
{\huge Problem Set Self-Assessment Guide}
\end{center}
For each problem...

If correct - congratulations! Keep up your hard work!

If incorrect try to determine what kind of error(s) you made, and make a note on the Canvas editor, and then over the next few days (prior to quiz) work on improving (see suggestions) when you either resolve problem set questions, or find similar questions in the textbooks.

\paragraph{{\Large{All Problems:}}}

\begin{itemize}
\item \textbf{Minor typo error:} Algebra errors, forgetting to carry terms from one equation to the next, etc. 
\begin{itemize}
\item If you are making this type of error, you should practice carefully checking each step of your work. It is easier to find these minor errors after taking a small break from the problem and looking at it with fresh eyes.
\end{itemize}
\item \textbf{Minor conceptual error:} You have the correct approach and are able to carry out that approach, but perhaps you forgot one step, or didn't account for an aspect of the problem. 
\begin{itemize}
\item If you are making this type of error, make sure you understand why what you did was an error and what you should have done (if not, come to office/tutoring hours). Before you start a problem, think about what errors you made on similar problems in the past, and consciously avoid.
\end{itemize}
\item \textbf{Correct approach but...:} This is when you have the correct approach but either you got stuck, or you made a more major error in carrying out that approach. 
\begin{itemize}
\item If you are making this mistake, look at the solution to the problem itself or to similar problems from class or textbooks, and make sure you understand the solution completely (if not, come to office/tutoring hours). Then, after taking a break, solve that problem again without looking at the solution. Compare your solution to the original and notice any places you differ, or if you got stuck again, think about what you should have done. Repeat until you can solve this type of problem on your own.
\end{itemize}
\item \textbf{Incorrect approach:} This is when you use an incorrect approach to solve a problem. (Note: there are sometimes multiple ways to approach the same problem, so if your solution is not exactly the same as mine, it might still be correct. If you are not sure, please contact your assigned TA.)
\begin{itemize}
 \item If you are making this mistake, follow the same approach as ``Correct approach but...'' to help you learn.
 \end{itemize} 
\item \textbf{Answered the wrong question:} 
\begin{itemize}
\item If you are making this mistake, read through each question several times and rephrase it in your own words before starting. Then, after you've started working, go back and read the question again and make sure you are still answering the question that was asked. After you finish, check again.
\end{itemize}
\end{itemize}


\paragraph{{\Large{Proof Problems:}}}

~\\Check for the above errors, and in addition, check for the following errors in your writing:
\begin{itemize}
\item \textbf{Not enough English:} Your proof consists of lines of equations with little/no explanation.
\begin{itemize}
\item  If you are making this error, after writing a proof, go back and insert more English as necessary.
\end{itemize}

\item \textbf{Math vs English:} You use English when mathematical notation would be significantly clearer, or vice versa. Note you often can interchange math and English, so this is only an issue when it makes a significant difference. 
\begin{itemize}
\item 
If you are making this mistake, after writing your proof, go back and look at places that seem convoluted either in English or math, and think about how you would translate that section. If it will be simpler the other way...translate!
\end{itemize}
\item \textbf{Mathematical Notation Usage:} You do not clearly introduce/explain your mathematical notation.
\begin{itemize}
\item  If you are making this error, after writing a proof, go back and insert more explanation as necessary.
\end{itemize}
\item \textbf{Writing style:} Your writing is too casual, or doesn't follow the norms of proof style writing.
\begin{itemize}
\item  If you are making this mistake, closely read through the solution proof or a similar proof from a textbook or class. After taking a break, rewrite the proof yourself without looking at the original. Compare and notice places where your language differs. Rewrite again trying to use more of the formal language.
\end{itemize} 
\item \textbf{Verbose:} You repeat yourself, explain too much.
\begin{itemize}
\item  If you are making this mistake, after writing a proof, go back edit, and cross out verbose passages.
\end{itemize}  
\end{itemize}



If you find that you are making a mistake that doesn't fall into the categories above, please let me know and I'll add to the list!


\vspace{1cm}

Your problem sets are graded based on effort. These self-grades are entirely to aid in your learning process and do not affect your grade in the course (except insofar as you put effort into the self-grade process). However, quizzes and tests are graded for correctness, and I
know some of you care a lot about grades, so I've included a rough correspondence between errors and grades on the following page. 

\newpage

The following are meant to give you a rough correspondence. Depending on the situation, the true grade may vary.


\begin{tabular}{|l|l|}
\hline
{\bf Error}& {\bf Grade}\\
\hline
  Correct &A (20/20)\\
  \hline 
  Minor typo error & A- (19/20)\\
  \hline
  Minor conceptual error &B+ (18/20)\\
  \hline
  Correct approach but...&B/B- (16-17/20)\\
  \hline
   Incorrect approach &C (15/20)\\
  \hline
  Answered the wrong question & wildcard!*  \\
  \hline
  No answer & :( why?  \\
  \hline
\end{tabular}
\bigskip



*depends on what question you answered and how well you answered it, but generally not more than A-

\paragraph{Proof Writing:} Roughly a part of a letter grade for each type of error (e.g. B+ $\rightarrow B$)

\end{document}