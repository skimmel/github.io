\documentclass[11pt]{article}

\usepackage[margin=1in]{geometry}
\usepackage{enumitem}
\usepackage[colorlinks]{hyperref}
\usepackage{url}\urlstyle{same}

\pagestyle{empty}
\setlength{\parindent}{0pt}
\setlength{\parskip}{3pt}

\begin{document}

% ~\vspace{-4em}
% \hrule\hrule\smallskip\centerline{\Huge{\bf DRAFT}}\smallskip\hrule\hrule
% \bigskip

\noindent{\Large {Reflection Worksheet}}

\bigskip
For each problem in the problem set, record your points, and write at least one sentence explaining why you think this problem was assigned. For example, is there a skill that the problem is helping you to practice? Or is there an aspect of a topic that the problem helps you to understand more fully? Or does this problem help you to achieve the course learning goals?
\begin{enumerate}
\item Points: $1/6$. {\it This problem gave us practice thinking through how a program logically proceeds step by step, especially in the case of if-else statements. }


\vspace{1cm}
\end{enumerate}

\vfill
Please write a paragraph reflecting on learning process for you in the course of both solving and grading the problem set. For example, what skills did the problem set help you to improve? are there are skills related to the problem set where you still need improvement? what concepts do you now understand, or don't yet understand? did you realize any misconceptions or repeated mistakes? what emotions did the problem set evoke (for example pride, frustration, confidence, confusion) and why?

\vspace{1cm}
{\it When I did this problem set, I didn't understand that if-else statements are like a true/false question where you do one thing if true (if part), and the other if false (else part). Now I understand that instead of having an ``otherwise'' decision flow symbol, I can use the false part of the if-else statements for ``otherwise.'' Even though I think my flow chart visually makes sense, it does not follow the step-by-step flow that a program follows. I think I
still need more practice thinking in this step-by-step fashion.}

\vspace{1cm}
\noindent Total Points: $1/6$

\end{document}